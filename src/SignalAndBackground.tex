%%%%%%%%%%%%%%%%%%%%%%%%%%%%%%%%%%%%%%%%%%%%%%%%%%%%%%%%%%%%


\subsection{Sources of backgrounds in NEXT} \label{sec:SignalAndBackground}
%%%

%%%%%%%%%%%
%\begin{figure}
%\centering
%\includegraphics[width=\textwidth]{img/TrackSignature.pdf}
%\caption{Monte Carlo simulation of signal (\bbonu\ decay of \Xe) and background (single electron of energy equal to the $Q$ value of \Xe) events in gaseous xenon at 15~bar. The ionization tracks left by signal events feature large energy deposits (or \emph{blobs}) at both ends.} \label{fig:TrackSignature}
%\end{figure}
%%%%%%%%%%

The relevance of any potential background source in NEXT depends on its probability to generate a signal-like track in the active volume of the detector with energy around the $Q$ value of \Xe. In principle, charged particles (muons, betas, etc.) entering the detector can be eliminated with essentially perfect efficiency defining a small veto region (of a few centimetres) around the boundaries of the active volume. Confined tracks generated by external neutral particles (such as high-energy gamma rays) or by internal contamination in the xenon gas can be suppressed taking advantage of both the excellent energy resolution of the detector and the topological signature.  


%%%%%%%%%%%%%%%%%%%%%%%%%%%%%%
\subsubsection*{High-energy gamma rays}
%%%
Natural radioactivity in detector materials and surroundings is, as in most other \bbonu-decay experiments, the main source of background in NEXT. In particular, the hypothetical \bbonu\ peak of \Xe\ ($\Qbb=2458.1\pm0.3$~keV lies in between the photo-peaks of the high-energy gammas emitted after the $\beta$ decays of \Bi\ and \Tl, intermediate products of the uranium and thorium series, respectively. 

The daughter isotope of \Bi, $^{214}$Po, emits a number of de-excitation gammas with energies around and above the $Q$ value of \Xe. Most of these gamma lines have very low intensity, and hence their contribution to the background rate is negligible. The gamma of 2447~keV (1.57\% intensity), however, is very close to \Qbb. Its photoelectric peak overlaps the signal peak even for energy resolutions as good as 0.5\% FWHM. The decay product of \Tl, $^{208}$Pb, emits a de-excitation photon of 2615~keV with an intensity of 99.75\%. Electron tracks from its photo-peak can lose energy via bremsstrahlung and fall in the \emph{region of interest} (ROI) around \Qbb\ defined by the energy resolution of the detector. Additionally, even though the Compton edge of the 2.6-MeV gamma is at 2382~keV, well below \Qbb, the Compton-scattered photon can generate other electron tracks close enough to the initial Compton electron to be reconstructed as a single track with energy around \Qbb. 

%%% TABLE %%%%%%%%%%%%%%%%%%%%
%%%%%%%%%%%%%%%%%%%%%%%%%%%%%%%%%%%%%%%%%%%%%%%%%%%%%%%%%%%%

%%%%%%%%%%
\begin{table}
\centering
{\small
\begin{tabular}{l l l l D{.}{.}{2.9} D{.}{.}{2.9}}
\toprule
%
Material & Subsystem & Technique & Units & \multicolumn{1}{l}{\Tl} & \multicolumn{1}{l}{\Bi} \\ \midrule
%
Copper (CuA1) & IS, EP, FC & GDMS & mBq/kg & <0.0014 & <0.012 \\
%
Fused silica & FC & NAA & mBq/kg & 0.0097(18) & 0.07(3) \\
%
Kapton board & TP & HPGe & mBq/unit & 0.0104(11) & 0.070(5) \\
%
Lead & OS & GDMS & mBq/kg & 0.034(7) & 0.35(7) \\
%
PMT R11410-10 & EP & HPGe & mBq/PMT & 0.30(9) & <0.94 \\
%
Polyethelene & FC & ICPMS & mBq/kg & <0.0076 & <0.062 \\
%
Resistor (1~G$\Omega$) & FC & HPGe & mBq/unit & 0.000011(6) & 0.00009(4) \\
%
Sapphire & EP & NAA & mBq/unit & 0.04(1) & <0.31 \\
%
Steel (316Ti) & PV & GDMS, HPGe & mBq/kg & <0.15 & <0.46 \\
%
SiPM SensL & TP & HPGe & mBq/unit & <0.00003 & <0.00009 \\
\bottomrule
\end{tabular} }
\caption{Specific activity of \Tl\ and \Bi\ in the most relevant materials and components used in the NEXT-100 detector \cite{Alvarez:2012as, Alvarez:2014kvs}. Three items (fused silica, sapphire and the field-cage resistors) have not been screened yet with sufficient precision; therefore, we use instead  measurements by the EXO Collaboration \cite{Leonard:2007uv, Auger:2012gs}. The activities determined via mass spectrometry (GDMS or ICPMS) or neutron activation analysis (NAA) were derived from Th and U concentrations. High-purity germanium (HPGe) $\gamma$-ray spectroscopy results correspond, whenever possible, to the lower parts of the natural decay chains. The figures in parentheses after the measurements give the 1-standard-deviation uncertainties in the last digits; the limits are given at 95\% CL. The abbreviations used to refer to the NEXT-100 detector subsystems have the following meaning: EP: energy plane; TP: tracking plane; FC: electric-field cage; PV: pressure vessel; IS: inner shielding; OS: outer shielding.} \label{tab:SpecificActivity}
\end{table}
%%%%%%%%%%

%%%%%%%%%%%%%%%%%%%%%%%%%%%%%%%%%%%%%%%%%%%%%%%%%%%%%%%%%%%% % \label{tab:SpecificActivity}
%%%%%%%%%%%%%%%%%%%%%%%%%%%%%%

%%% TABLE %%%%%%%%%%%%%%%%%%%%

%%%%%%%%%%
\begin{table}[!]
\centering
\begin{tabular}{lll c c}
\toprule
Detector subsystem & Material & Quantity & \multicolumn{1}{c}{\Tl} & \Bi\ \\ 
                   &          &          & \multicolumn{1}{c}{(mBq)} & (mBq) \\ \midrule
%
\emph{Pressure vessel} \\
\quad Total & Steel 316Ti & 1310~kg & $<197$ & $<603$ \\ \addlinespace
%
\emph{Energy plane} \\
%
\quad PMTs & R11410-10 & 60~units & $12(3)$ & $<56$ \\
\quad PMT enclosures & Copper CuA1 & 60$\times$4.3 kg & $<0.36$ & $<3.1$ \\
\quad Enclosure windows & Sapphire & 60$\times$0.14 kg & $0.34(8)$ & $<2.6$ \\ 
\quad Support plate & Copper CuA1 & 408~kg & $<0.6$ & $<5$ \\ \addlinespace 
%
\emph{Tracking plane} \\
%
\quad SiPMs & {\scshape Sensl} 1~mm$^{2}$ & 107$\times$64 units & $<5$ & $<18$ \\
\quad Boards & Kapton FPC & 107 units & $1.5(2)$ & $3.2(1.1)$ \\ \addlinespace 
%
\emph{Field cage} \\
%
\quad Barrel & Polyethylene & 128~kg & $<1$ & $<8$ \\
\quad Shaping rings & Copper CuA1 & 120$\times$3~kg & $<0.5$ & $<4$ \\
\quad Electrode rings & Steel 316Ti & 2$\times$5~kg & $1.5$ & $<5$ \\
\quad Anode plate & Fused silica & 9.5~kg & $0.092(17)$ & $0.7(3)$ \\
\quad Resistor chain & 1-G$\Omega$ resistors & 240 units & $<0.0026$ & $<0.020$ \\ \addlinespace
%
\emph{Shielding} \\
%
\quad Inner shield & Copper CuA1 & 9210~kg & $<13$ & $<111$ \\
\quad Outer shield & Lead & 60700~kg & $2060(430)$ & 21300(4300) \\
\bottomrule
\end{tabular}
\caption{Radioactivity budget of the NEXT-100 detector. The figures in parentheses after the measurements give the 1-sigma uncertainties in the last digit. The upper limits in the activity of most subsystems originate in the 95\% CL limits set on the specific activity of the corresponding materials quoted on Table~\ref{tab:SpecificActivity}.} \label{tab:RadioactiveBudget}
\end{table}
%%%%%%%%%% % \label{tab:RadioactiveBudget}
%%%%%%%%%%%%%%%%%%%%%%%%%%%%%%

The NEXT Collaboration is carrying out a thorough campaign of material screening and selection using gamma-ray spectroscopy (with the assistance of the LSC Radiopurity Service) and mass spectrometry techniques (ICPMS and GDMS). Table~\ref{tab:SpecificActivity} collects the measurements of the specific activity of \Tl\ and \Bi\ in the most relevant materials and components used in the NEXT-100 detector, and Table~\ref{tab:RadioactiveBudget} details the radioactivity budget of NEXT-100 separated into detector subsystems.

The rock walls of the underground laboratory are a rather intense source of high-energy gammas due to the presence of trace radioactive contaminants in their composition. The total gamma flux in Hall A at LSC is $1.06\pm0.24$~cm$^{-2}$~s$^{-1}$, with contributions from $^{40}$K ($0.52\pm0.23$~cm$^{-2}$~s$^{-1}$), $^{238}$U ($0.35\pm0.03$~cm$^{-2}$~s$^{-1}$) and $^{232}$Th ($0.19\pm0.04$~cm$^{-2}$~s$^{-1}$). Nevertheless, the external lead shield of NEXT-100 will attenuate this flux by more than 4 orders of magnitude, making its contribution to the final background rate negligible. 

%%%%%%%%%%%%%%%%%%%%%%%%%%%%%%
\subsubsection*{Radon}
%%%

The measured activity of airborne radon ($^{222}$Rn) at the Laboratorio Subterr\'aneo de Canfranc (Hall A) varies between 60 and 80~mBq~m$^{-3}$. Left at this level, radon would represent an intolerably high source of gamma rays from \Bi. For this reason, the vicinity of the detector (the internal volume of the lead castle shield) will be flushed with clean air produced by a radon mitigation system such as those used, for instance, by the NEMO-3 and DarkSide experiments. A reduction of, at least, a factor of 100 in the activity of airborne radon is expected.

Radon can also emanate from detector components and be transported to the active volume through the gas circulation. The $\alpha$ decays of radon (either $^{220}$Rn or $^{222}$Rn) in the bulk xenon do not represent a background: they have energies well above \Qbb\ and their very short tracks are easily identified \cite{Alvarez:2012hu}. The progeny of radon is positively charged and will drift toward the TPC cathode. A majority of the subsequent \Bi\ and \Tl\ beta decays will occur on the cathode rather than in the active volume. These cathode events are equivalent to other background sources close to the active volume (\Tl\ and \Bi\ decays from the sensor planes, for instance): if the $\beta$ particle enters the active volume, the event can then be vetoed; otherwise, the de-excitation gamma rays that interact in the xenon can generate background tracks. In addition, a small fraction (0.2\%) of the \Bi\ $\beta$ decays occurring in the xenon bulk will produce an electron track with energy around \Qbb. Luckily, the disintegration of \Bi\ is followed shortly after by the $\alpha$ decay of $^{214}$Po ($T_{1/2} = 164~\mu$s \cite{Wu:2012nds}). The detection of this so-called Bi-Po coincidence can be used to identify and suppress with high efficiency these background events.

The design of NEXT-100 minimizes the use of materials and components known to emanate radon in high rates, such as plastics, cables or certain seals. Nevertheless, estimating a priori the emanation rate and radon activity in the xenon is difficult, since the available data are scant and have been acquired in very different conditions (in vacuum, typically). Understanding the impact of radon emanation and its suppression---by means of a radon trap, for example---is one of the priorities of the NEW program (see section \ref{sec.new}).
