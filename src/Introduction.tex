
The purpose of this document is to present the status and the prospects of a technology for \bbonu\ searches based in a High Pressure Xenon (HPXe) TPC with electroluminescent readout. Such a technology presentes the following advantages:

\begin{enumerate}
\item {\bf It uses xenon enriched at 90\% in the \XE\ isotope}: xenon is one of the most interesting \bb\ isotopes, given its (comparatively) low cost. Xenon costs around 1.3 \$ per gram at the time of writing this report. One ton of natural xenon may cost around 1.3 M\$, and one ton of enriched xenon less than 15 M\$. Indeed, three experiments own today enriched xenon: NEXT-100 (100 kg), EXO ($\sim$ 200 kg) and KamLAND-Zen ($\sim$ 800 kg). Xenon is, therefore, the only of the \bb\ isotopes which already deploys more than one ton of enriched material. 
\item {\bf It is scalable}: xenon is an excellent active target, and the bulk of the interactions in a TPC occur in a fiducial region, away from surfaces, thus making it possible the active veto of surface-related background such as alpha particles. Assuming for simplicity the detector to be a box of dimension $L$, the needed instrumentation, as well as the main backgrounds scale roughly with $L^2$, while the target mass scales with $L^3$. The signal-to-noise ratio, then, should improve linearly with $L$. 
\item {\bf Energy resolution}: an HPXe-EL TPC deploys an excellent energy resolution due to the use of proportional electroluminescence\cite{Nygren:2009zz}. As further discussed below, the target for the technology is a resolution of 0.5\% FWHM at \Qbb.
\item {\bf Topological signature}: an HPXe operating at a pressure of 10-20 bar offers a {\em topological signature}, e.g, the clear definition of the signal as an isolated, ``single track'' featuring two high energy depositions, ``blobs'', near the track extremes. Such track is required to be fully contained in the fiducial region with no other energy depositions in the chamber. 
\end{enumerate}

An HPXe can be built with radiopure materials and be shielded from external radioactivity. Radon suppression techniques can also control the effect of radon-related backgrounds. The technology, then, presents many attractive features. In particular, the energy resolution is considerably better than in liquid xenon (EXO achieves 3.5 \% FWHM  at \Qbb using anti-correlation between the scintillation and the ionisation signal) and much better than in xenon dissolved in liquid scintillator (KamLAND-ZEN resolution is around 10\% FWHM at \Qbb). The availability of the topological signature is also unique of the HPXe, while the scalability and the benefits of an active target are common advantages with the other two techniques. The weakest points of an HPXe compared with liquid xenon and xenon dissolved in scintillator are:

\begin{enumerate}
\item  The density of the target material is lower, and thus the dimensions of the TPC are larger (than, for example, the equivalent liquid xenon TPC). An apparatus holding one ton of xenon at 20 bar would require a volume of $\sim 10^3$~m, implying longitudinal dimensions in the range of 3 m. This appears feasible a priory, but must be demonstrated.
\item  The overall selection efficiency is lower than for the other xenon-based detectors, due in part to bremsstrahlung depleting the \Qbb\ peak, and in part to the efficiency cost of the topological signature. 
\end{enumerate}

To demonstrate the suitability of the technology as a candidate for a ton-scale detector for \bbonu\ searches, it is necessary to:

\begin{enumerate}
\item Demonstrate the robustness and the scalability of the technology.
\item Demonstrate excellent energy resolution.
\item Maximise the discriminating power of the topological signature.
\item Demonstrate from the data themselves the very low background level predicted by Monte Carlo calculations. 
\end{enumerate}

All the above points can be addressed in the forthcoming 3 years by the NEXT collaboration~
 \cite{Granena:2009it,Alvarez:2012sma,Gomez-Cadenas:2014dxa}.
 % as described in this document, which is organised as follows: 
%
%\begin{enumerate}
%\item {\bf Late 2015 to mid 2017:} after successful operation of 1-kg non-radiopure prototypes, the collaboration is currently commissioning the NEW detector, a 10 kg, radiopure prototype, which will start data taking in late 2015 or early 2016. NEW represents an order of magnitude increase of target mass with respect to the prototypes and will be operating underground at the LSC (Canfranc underground laboratory, in Spain) during a period of 18-24 months. NEW operation should allow the confirmation of the excellent energy resolution measured by NEXT-DEMO and NEXT-DBDM prototypes\cite{Alvarez:2012kua,Alvarez:2012xda,Alvarez:2013gxa} . It is also a key tool to investigate how to maximise the discriminating power of the topological signature. It will also allow to assess the radio purity of the technology (and eventually to improve it). 
%\item {\bf Mid 2017 to end of 2018 and beyond:} The collaboration plans to assemble the NEXT-100 detector in 2017 (while still taking data with NEW) and operate it in late 2017, then 2018 and beyond. NEXT-100 should apply the lessons learned in NEW and be a full demonstrator of the HPXe technology and its capability to be extrapolated to the ton scale in addition of being a leading \bbonu\ experiment on its own right.  
%\end{enumerate}
%
