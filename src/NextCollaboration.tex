
 \subsection{Organisation of the collaboration}
 
NEXT is an international collaboration including groups from Spain, Portugal, Russia, Colombia and the USA. The project has been headed from its very beginning by a Spanish physicist (J.J. Gomez-Cadenas) and an american physicist (Dave Nygren). The contribution of Dave Nygren (longtime at LBNL, now at University of Texas at Arlington, UTA), Azriel Goldschmidt (LBNL), the late James White (Texas A\&M), Robert Webb (Texas A\&M), and John Hauptman (Iowa State), among others have been essential for the success of the R\&D phase, epitomised by the DEMO and DBDM detectors, the design of the detector, summarised in our TDR \cite{Alvarez:2012sma} and the construction of NEW. Furthermore, NEXT software uses the {\em art} framework, a product developed and maintained by Fermilab, and several Fermilab physicists (notably Adam Para and Paul Lebrun) are currently leading, together with an american Fulbright fellow (Josh Renner) the development of improved reconstruction and pattern recognition algorithms. 

The strong influence of the USA collaboration is recognised explicitly by the nomination of Dave Nygren as International Spokesperson of NEXT and the charge that such role implies. Specifically, while the role of the Spokesperson is to ensure the short-term plans of the collaboration (e.g, the construction, operation and the search for \bbonu\ events of the NEXT-100 detector), the role of the International Spokesperson is to promote the technology as a candidate to explore the IH, as well as seeking to enlarge internationally (and in particular in the USA) the collaboration. 

\subsection{Funding}

NEXT has three sources of funding:
\begin{enumerate}
\item {\bf Funding provided by the Spanish secretary of state for science (SEIDI)}. Spain, as host country of the NEXT-100 experiment, provides the facilities of the Canfranc Underground Laboratory, and has funded the NEXT project during the last seven years, with a total contribution approaching 10 M \euro. 
\item {\bf Funding provided by the European Research Council}, ERC, through an Advanced Grant ($\sim 3$M \euro) awarded to Gomez-Cadenas in 2014.
\item {\bf Funding provided by the international collaboration}. All the NEXT groups contribute to common fund. In addition, the USA contingent has made substantial contributions to the project, both in terms of man power and equipment. 
\end{enumerate}


\subsection{Responsibilities}

The collaboration is currently starting to commission the NEW apparatus. The detector has been fully payed, including a substantial contribution from UTA, which has purchased the SiPMs of the tracking plane and part of the DAQ electronics. The operation and analysis of the detector, as well as the R\&D program (gas mixtures studies with DEMO and NEW, and the operation of DEMO at CERN), require additional contributions from the international collaboration. 

In particular, the USA groups have the following responsibilities in NEW:

\begin{enumerate}
\item F. Monrabal (a UTA postdoctoral associate) is the project leader of the NEW field cage and is in charge of its installation (foreseen for September), commissioning and operation. Monrabal will also be leading the studies with gas mixtures in DEMO.
\item R. Webb (Texas A\&M) has built the electroluminescent amplification system of NEW (EL grid, quartz plate), which is currently being delivered to the LSC. 
\item The Fermilab group is in charge of the general support of the {\em art} framework and lead the development of reconstruction and pattern recognition algorithms (together with Josh Renner).
\item Josh Renner (a LBNL Ph.D., currently a Fulbright Fellow working at IFIC) is leading the studies related with the magnetic field, and will be leading the experimental campaign at CERN.
\end{enumerate}

The NEW effort requires man power from the international collaboration to commission, run and analyse the data.
In particular the contribution of the USA groups is essential for the success of the project. Specifically we propose the following responsibilities for the USA contingent:

\begin{enumerate}
\item {\bf Design engineering}: the design of the NEXT-100 detector was lead by D. Schuman, a senior engineer, then at LBNL. The construction of NEW has followed closely the solutions proposed by the Schuman design, but the construction of NEXT-100, a larger and more complex detector, requires of additional work to review and certify some of the most delicate issues, such as the High Voltage Feedthroughs, the vacuum system of the energy plane, or the design of the field cage (which can be improved to make it more radiopure). 
\item {\bf Energy plane}: The PMTs of NEXT-100 need to operate in vacuum, protected by individual copper enclosures (cans) which are continuously pumped. The construction of such a system requires strong engineering capabilities. We would like the USA groups to take responsibility of the construction and commissioning of the energy plane cans and associated pumping system.  
\item {\bf Field cage}: the field cage is one of the most challenging parts of the detector. Its design was led by Nygren and the late James White, and the construction of the NEW TPC has been lead by F. Monrabal, currently at UTA. We expect that the USA contingent take the full responsibility for the construction, commissioning and operation of the NEXT-100 field cage. 
\item {\bf Software, analysis and data processing}: We would like to strengthen and amplify the current collaboration with Fermilab. In addition to using {\em art} and to the contribution of Fermilab physicists to reconstruction and software, we would like to be able to process Monte Carlo data (huge productions are needed to simulate all the relevant backgrounds) at Fermilab. 
\end{enumerate}

The other key elements of the detector are the pressure vessel, gas system, tracking plane, PMT and SiPM calibration, front-end electronics, DAQ and slow control. Most of these system will be constructed by the Spanish groups, under the leadership of the spokesperson and with help from other international groups. 

Two essential roles in any experiment, in particular during the construction and commissioning phase are that of the Project Manager (PM) and the Technical Coordinator (TC). To reflect the weight of USA in the project, the TC will be taken, starting in 2016, by the UTA group. 
 
\subsection{Time schedule}
\begin{itemize}
\item {\bf 2015:} installation of NEW, start of the R\&D campaign to study gas mixtures with NEXT-DEMO.
\item {\bf 2016:} commissioning and operation of NEW with natural xenon. Studies of energy resolution, electron reconstruction and topological signature. Complete R\&D campaign to study gas mixtures with NEXT-DEMO. Full review of NEXT-100 design. Full test of gas system.
\item {\bf 2017:} Construction of NEXT-100 detector parts: Pressure Vessel and gas system are already in place. The major systems to be built are the energy plane, tracking plane, field cage and inner copper shielding. Installation of NEXT-100 at the LSC.
\item {\bf 2017:} commissioning and operation of NEXT-100 with natural xenon. Full calibration of detector and start of the physics campaign. Operation of NEW with suitable gas mixtures to demonstrate in a large-scale detector the improvement of the topological signature. Test of magnetic field using DEMO at CERN. 
\end{itemize}

\subsection{Man power resources from the USA}

The following list defines a full time scientist (FTS), as a post-doctoral associate, research associate or assistant professor (with no teaching obligations, or the corresponding fraction if teaching obligations are taking into account). FTS do not include typically full professors or senior staff members at universities or national laboratories. A full time engineer (FTE) is defined as a senior engineer who devotes 100\% of his or her time to the project. A Graduate Student (GS) is defined as a USA scientist working for his or her Ph.D. thesis in the NEXT experiment. This list includes resources related with the construction, commissioning and installation (CCI) of the NEW and NEXT-100 detectors, but does not include resourced devoted to the development of software and analysis, where we assume a continuing collaboration with Fermilab. 

\begin{itemize}
\item {\bf 2016:} One FTS, leading the operation of NEW (pure xenon). One FTE working in the NEXT-100 design review. One GS involved in the operation and data analysis of NEW. 
\item {\bf 2017:} Three FTS, one leading the operation of NEW (gas mixtures), one leading the CCI of the NEXT-100 field cage (FC) at UTA, one leading the CCI of the NEXT-100 energy plane (EP) cans and vacuum system (at UTA or elsewhere). One FTE, sharing his or her time between the construction of the FC and the EP. Workshop time and technician man power for the construction of FC and EP.  Two additional GE. One involved in the FC project, the other in the EP project. 
\item {\bf 2018:} Three FTS, involved in the analysis of NEXT-100, the completion of the R\&D (in particular concerning magnetic field) and in the design of a ton-scale HPXe-EL detector.  One FTE, leading the engineering design of the ton-scale detector. One additional GE working in the simulation and design studies of the ton-scale detector as well as in the analysis of NEXT-100 data.
\end{itemize}

\subsection{In-kind contribution from the USA}

\begin{itemize}
\item {\bf Common fund for 2016-2018}: (20 k\$ a year).  
\item {\bf Pays the construction, transport and installation of field cage:} estimated cost 
250 k\$
\item {\bf Pays the construction, transport and installation of energy plane mechanics:} estimated cost 250 k\$.
\item {\bf Contributes to the costs of infrastructures}: 200 k\$.
\item {\bf Contributes to the costs of DAQ and computing}: 150 k\$.
\end{itemize}

In total the contribution of the USA groups for 2016-2018 is estimated in 910 k\$. 
 
